%-------------------------------------------------------------------------------
%	SECTION TITLE
%-------------------------------------------------------------------------------
\cvsection{Projects}

%-------------------------------------------------------------------------------
%	CONTENT
%-------------------------------------------------------------------------------
\begin{cventries}
%-----------------------------------------------------------
%   \cventry
%     {8th Inter IIT Tech Meet, IIT Roorkee} % Role
%     {DRDO SASE's UAV Fleet Challenge} % Event
%     {\hspace{-15em} \href{https://github.com/AerialRobotics-IITK/inter\_iit\_uav\_fleet}{AerialRobotics-IITK/inter\_iit\_uav\_fleet}}% Location
%     {Nov-Dec. 2019: \href{www.interiittech.org}{interiittech.org}} % Date(s)
%     {
%       \begin{cvitems} % Description(s)
%         \item Won the \textbf{sole Gold Medal} amongst \textbf{\textit{20}} participating IITs, achieving a \textbf{perfect score} (\textit{400/400}) and a special mention
%         \item Developed a \textbf{fleet of MAVs} capable of \textbf{collaboratively surveying} a \textit{1600 sq.m.} grass field to locate green boxes
%         \item Implemented a finite-state machine for \textbf{high-level planning} of the complete survey mission flow using Boost MSM
%         \item Extracted the \textbf{target GPS coordinates} by implementing HSV-thresholded \textbf{box-contour detection} using OpenCV and subsequent \textbf{pose transformations} based on the pinhole camera model, achieving \textbf{centimeter-level precision}
%         \item Setup a multimaster architecture for \textbf{intra-fleet communication} to accurately track number of detected targets
%       \end{cvitems}
%     }

%---------------------------------------------------------
%   \cventry
%     {Outdoor Problem Statement --- Team Aerial Robotics IITK} % Role
%     {International Micro Aerial Vehicle Challenge 2019, Madrid, Spain} % Event
%     {\hspace{-5em} \href{https://github.com/AerialRobotics-IITK/imav2019}{AerialRobotics-IITK/imav2019}} % Location
%     {Mar-Oct. 2019: \href{www.imav2019.org}{imav2019.org}}% Date(s)
%     {
%       \begin{cvitems} % Description(s)
%         \item Amongst the \textbf{top 15 student teams internationally} to qualify for the competition, \textit{only} student team from India
%         \item Researched optimal mission planning for \textbf{coordinated surveying} of a \textit{30000 sq.m}. area using a \textbf{fleet of 3 MAVs}
%         \item Implemented a \textbf{floodfill-based vision algorithm} for the detection of multiple and differently coloured mailboxes
%         \item Applied \textbf{visual servoing} on the detection feedback for the delivery of packages into the detected mailboxes
%         \item Engineered a \textbf{publish-subscribe messaging system} for intrafleet sharing of delivery status and drop locations
%         \item Designed the \textbf{state transition diagram} for the complete mission plan and implemented it within an FSM
%         \item Explored the use of \textbf{feature detection algorithms} for the detection of a \textbf{crashed MAV} within the surveying region
%         \item Implemented a \textbf{CNN framework} for crashed MAV detection task based on \textbf{YOLOv3} using drone image datasets
%       \end{cvitems}
%     }
    
%---------------------------------------------------------
  \cventry
    {Summer Project, Programming Club IITK} % Role
    {ML @ Kaggle} % Event
    {\hspace{-5em} } % Location
    {May 2020 -July. 2019} % Date(s)
    {
      \begin{cvitems} % Description(s)
        \item A project aimed at learning the basics and higher levels of machine learning through competitions on \textbf{Kaggle}
        \item Used PyTorch for building the models from scratch at first
        \item In the later part did sentiment analysis and Word Embeddings using Word2Vec 
      \end{cvitems}
    }

%---------------------------------------------------------
%  \cventry
%     {Self Project --- \emph{Aerial Robotics IITK}} % Role
%     {Planning Library} % Event
%     {\hspace{-25em} \href{https://github.com/AerialRobotics-IITK/ariitk\_planning\_library}{AerialRobotics-IITK/ariitk\_planning\_library}} % Location
%     {Apr-May. 2020} % Date(s)
%     {
%       \begin{cvitems} % Description(s)
%         \item Studied the generation of \textbf{Truncated and Euclidean Signed Distance Fields} from depth pointclouds using Voxblox
%         \item Experimented with the \textbf{generation of sparse 3D graphs} by skeletonizing the Voronoi diagram of the ESDF
%         \item Implemented a \textbf{global planner using A*} on the sparse graph for \textbf{real-time path planning} in a known environment
%         \item Built a \textbf{local planner for inter-waypoint planning} by using A* on a graph generated by randomly sampling points
%         \item Explored frontier-based navigation for \textbf{autonomous exploration of unknown environments} using TSDFs
%       \end{cvitems}
%     }
    
%---------------------------------------------------------
%  \cventry
%     {Summer Project Mentor --- \emph{Robotics Club IITK}} % Role
%     {Autonomous Navigation in Rough Terrain Environments} % Event
%     {\hspace{-25em} \href{https://github.com/RoboticsClubIITK/2020\_rover}{RoboticsClubIITK/2020\_rover}} % Location
%     {May-Jul. 2020} % Date(s)
%     {
%       \begin{cvitems} % Description(s)
%         \item Mentored a team of \textit{15} students in implementing a complete \textbf{navigation stack} on a \textbf{six-wheeled rover} in simulation
%         \item Built the \textbf{simulation environment} for the project in Gazebo using open-source terrain models and rover URDFs 
%         \item Configured servo motors, a 2D \textbf{LiDAR}, \textbf{depth camera} and a GPS sensor via the use of \textbf{ROS controllers} and plugins
%         \item Guided the implementation of GPS and \textbf{position controllers}, \textbf{fiducial marker detection}, LiDAR-based \\ \textbf{obstacle avoidance} and \textbf{path planning} for the rover
%       \end{cvitems}
%     }
    
%-------------------------------------------------------- 
%   \cventry
%     {Summer Workshop, Brain and Cognitive Society} % Role
%     {Basics of Reverse Engineering the Brain} % Event
%     {\hspace{-15em} \href{https://github.com/ashwin2802/BCS\_Workshop\_Apr\_20}{ashwin2802/BCS\_Workshop\_Apr\_20}} % Location
%     {Apr. 2020} % Date(s)
%     {
%       \begin{cvitems} % Description(s)
%         \item Evaluated various \textbf{CIFAR10 image classification architectures} and implemented an \textbf{RL agent using Q-learning}
%         \item Studied and implemented Atkinsion-Shiffrin's \textbf{Multi-Store Model of Memory} and verified the \textbf{serial position effect}
%         \item Verified the postulates of Feature Integration Theory's \textbf{Model of Attention} by modelling feature search in images
%         \item Attempted to \textbf{replicate the results of research} on \textit{the role of categorization in visual search for orientation} by \\ \textbf{conducting a survey} generated using Psytoolkit and statistically analysing the data from \textit{10} participants
%       \end{cvitems}
%     }

%---------------------------------------------------------
%   \cventry
%     {Course Project, EE683 --- \emph{Supervisor: Prof. Shilpi Gupta}} % Role
%     {Simulation of a Beam Splitter} % Event
%     {\hspace{-10em} \href{https://github.com/ashwin2802/EE683}{ashwin2802/EE683}} % Location
%     {Oct-Nov. 2019} % Date(s)
%     {
%       \begin{cvitems} % Description(s)
%         \item Modelled \textbf{single photon sources} in Python with uniform, exponential and logarithmic emission distributions
%         \item Derived expressions for the \textbf{beam splitter operator} applying basic principles of quantum mechanics and optics
%         \item Implemented the derived expressions to \textbf{verify experimental results and postulates} such as the \textit{Hong-Ou-Mandel effect} by simulating various input configurations and computing the \textbf{second-order output correlation}
%       \end{cvitems}
%     }
    
%--------------------------------------------------------
%   \cventry
%     {Course Project, TA201 --- \emph{Supervisor: Prof. Anish Upadhyaya}} % Role
%     {Bar Pantograph} % Event
%     {} % Location
%     {Aug-Nov. 2019} % Date(s)
%     {
%       \begin{cvitems} % Description(s)
%         % \item Awarded \textbf{Best Sectional Project} for timely project completion and work product quality out of \textit{\textbf{20+}} projects
%         \item Designed a simple parallelogram bar mechanism \textbf{prototype that could copy designs} drawn by the operator 
%         \item Demonstrated proof-of-concept via \textbf{AutoCAD assembly} and built the prototype working in a \textit{team of 5}
%         \item Acquired \textbf{hands-on experience} of basic manufacturing processes like \textit{welding, brazing, casting and forging}
%       \end{cvitems}
%     }
  
%--------------------------------------------------------
  \cventry
    {Summer Project, Programming Club IITK} % Role
    {Basic CP} % Event
    {\hspace{-5em} } % Location
    {May 2020 -July. 2019} % Date(s)
    {
      \begin{cvitems} % Description(s)
        \item Aimed to learn how to efficiently solve \textbf{Competitive} \textbf{Programming} questions
        \item The Project aimed to introduce and develop Programming skills and how to code with less memory and time complexity
      \end{cvitems}
    }

%---------------------------------------------------------
\cventry
    {Hackathon, Programming Club IITK} % Role
    {Deep Dive Into Hackathons (NLP)} % Event
    {\hspace{-5em} } % Location
    {August 2020 - } % Date(s)
    {
      \begin{cvitems} % Description(s)
        \item A series of deep Learning Hackathons
        \item Currently participating in the hackathon \textbf{\href{https://www.kaggle.com/c/gendered-pronoun-resolution}{Gendered Pronoun Resolution}} on \textbf{Kaggle}
      \end{cvitems}
    }
    
%---------------------------------------------------------
%   \cventry
%     {Summer Project, Programming Club IITK} % Role
%     {Introduction to Quantum Computing} % Event
%     {IIT Kanpur, India} % Location
%     {Aug-Nov. 2019} % Date(s)
%     {
%       \begin{cvitems} % Description(s)
%         \item blah blah
%         \item blah blah
%         \item blah blah
%       \end{cvitems}
%     }

%-------------------------------------------------------- 
%   \cventry
%     {Team Aerial Robotics IITK} % Role
%     {IARC 2020} % Event
%     {IIT Kanpur, India} % Location
%     {Apr. 2020-Present} % Date(s)
%     {
%       \begin{cvitems} % Description(s)
%         \item blah blah
%         \item blah blah
%         \item blah blah
%       \end{cvitems}
%     }

%-------------------------------------------------------- 
% \cventry
%     {Team Aerial Robotics, IITK} % Job title
%     {Aerial Surveillance} % Organization
%     {IIT Kanpur, India} % Location
%     {Jan. 2020 - Present} % Date(s)
%     {
%       \begin{cvitems} % Description(s) of tasks/responsibilities
%         \item Analyzed the accuracy and robustness of CNN frameworks YOLOv3 and YOLOv3-tiny for detection of person-like objects by validating them on various surveillance datasets
%         \item Researched the reliability of SORT and Deep-SORT algorithms for the real-time tracking of multiple objects detected in an image frame
%         \item Applied transfer learning to a network trained on fixed-camera surveillance using an annotated aerial surveillance dataset.
%       \end{cvitems}
%     }
%---------------------------------------------------------
\end{cventries}